\documentclass{article}

\usepackage[T1]{fontenc}            % Codificação para português 
\usepackage[portuguese]{babel}      % Português
\usepackage{hyphenat}               % Use hifens corretamente
\usepackage{fancyhdr}               % Header estilo DCC
\usepackage{enumerate}
\usepackage{listings}
\usepackage{xcolor}

\definecolor{codegreen}{rgb}{0,0.6,0}
\definecolor{codegray}{rgb}{0.5,0.5,0.5}
\definecolor{codepurple}{rgb}{0.58,0,0.82}
\definecolor{backcolour}{rgb}{0.95,0.95,0.92}

\lstdefinestyle{mystyle}{
    backgroundcolor=\color{backcolour},   
    commentstyle=\color{codegreen},
    keywordstyle=\color{magenta},
    numberstyle=\tiny\color{codegray},
    stringstyle=\color{codepurple},
    basicstyle=\ttfamily\footnotesize,
    breakatwhitespace=false,
    breaklines=true,                 
    captionpos=b,                    
    keepspaces=true,                 
    numbersep=5pt,                  
    showspaces=false,                
    showstringspaces=false,
    showtabs=false,                  
    tabsize=2
}

\lstset{style=mystyle}

\setlength{\headheight}{24pt} % O LaTeX reclama sem isso

\makeatletter
\renewcommand{\maketitle}{%
    \begin{center}
        {\LARGE\@title\par}
        \vskip 0.5em
        {\large\begin{tabular}[t]{c}\@author\end{tabular}\par}
        \vskip 1em
        {\large\textbf{\@date}\par}
        \vskip 1em
    \end{center}
}

\title{Universidade Federal de Minas Gerais

Ciência da Computação}
\author{Igor Lacerda Faria da Silva}
\date{Lista de Exercícios 4}

\begin{document}

\pagestyle{fancy}
\lhead{DCC024 Linguagens de Programação}
\rhead{2023.1}

\maketitle

\begin{enumerate}

    \section*{Passagem de Parâmetros}

    \item \begin{enumerate}
            \item É impresso o número 5.
            \item Será impresso o número 7, pois a variável \texttt{y} e a variável \texttt{z} apontariam para o mesmo endereço. Dessa maneira, a variável global \texttt{x} manteria sua contribuição de 1 para a soma, mas o valor comum da referência dos parâmetros seria incrementado duas vezes.
        \end{enumerate}
    \item Não, eu não diria que é um mecanismo bastante utilizado em C
        \begin{enumerate}
            \item
\begin{lstlisting}[language=C]
#include <stdio.h>

int main(int argc, char **argv) {
    printf("sum = %d\n", (argc) + (argv[0][0]));
}
\end{lstlisting}
            \item A captura de variáveis ocorre quando uma macro recebe dois ou mais argumentos iguais e, durante a expansão, as variáveis são `misturadas', o que pode causar um comportamento indesejado.
            \item 
\begin{lstlisting}[language=C]
#include <stdio.h>

#define LIGMA(X) (X) = (X) + 1, (X) = 2 * (X)

int main(void) {
  int x = 0;
  printf("ligma = %d", LIGMA(x));
}
\end{lstlisting}

        \end{enumerate}
        \item 
            \begin{enumerate}
                \item 30
                \item O programa lê um um inteiro $a$ da entrada e calcula a soma do quadrado das frações $ \frac{1}{i+a} $ com $i$ variando de 1 até 100.
            \end{enumerate}
        \item 
            \begin{enumerate}
                \item m1: 4, m2: 3
                \item m1: 4, m2: 3
                \item m1: 4, m2: 4
                \item Passagem por referência.
                \item Passagem por cópia.
            \end{enumerate}

    \end{enumerate}

    \section*{Programação Lógica}

    \begin{enumerate}
        \item 
            \begin{enumerate}
                \item 
\begin{lstlisting}[language=prolog]
firstCousin(X, Y) :-
  parent(P1, X),
  parent(P2, Y),
  P1 \= P2,               % X e Y têm pais diferentes
  \+ sibling(X, Y).       % X e Y não são irmãos
\end{lstlisting}    

                \item 
\begin{lstlisting}[language=prolog]
descendant(X, Y) :- parent(Y, X).
descendant(X, Y) :- parent(Z, X), descendant(Z, Y).
\end{lstlisting}
            \end{enumerate}

            \item 
\begin{lstlisting}[language=prolog]
third(X, Y) :- X = [_, _, Y|_].
\end{lstlisting}

            \item 
\begin{lstlisting}[language=prolog]
dupList([], []).
dupList([H|T], [H, H|X]) :- dupList(T, X).
\end{lstlisting}

            \item 
\begin{lstlisting}[language=prolog]
isEqual([], []).
isEqual(X, Y) :- permutation(X, Y).
\end{lstlisting}

            \item 
\begin{lstlisting}[language=prolog]
isDifference([], _, []).
isDifference([X|Xs], Y, Z) :-
  member(X, Y),
  isDifference(Xs, Y, Z).
isDifference([X|Xs], Y, [X|Z]) :-
  not_member(X, Y),
  isDifference(Xs, Y, Z).

not_member(_, []).
not_member(X, [Y|Ys]) :-
  X \= Y,
  not_member(X, Ys).
\end{lstlisting}
            \item 
                \begin{enumerate}[I]
                    \item 
                        Assumindo a negação da cláusula a ser demonstrada: 

                        $\sim$ (append(X, Y, [1, 2]))
                    \item Aplicando a negação na primeira cláusula do predicado append:

                        $\sim$ (append([], Y, [1, 2]))
                    \item Aplicando a negação na segunda cláusula do predicado append:

                        $\sim$ (append([Head|TailA], Y, [1, 2])) :- $\sim$ (append(TailA, Y, TailC))
                    \item 
                        Aplicando a negação na primeira cláusula do predicado append novamente:

                        $\sim$ (append([Head|TailA], Y, [1, 2])) :- $\sim$ (append(TailA, Y, TailC)), $\sim$ (Head = 1), $\sim$ (TailC = [2])
                    \item 
                        Aplicando a negação em Head = 1:

                        $\sim$ (append([Head|TailA], Y, [1, 2])) :- $\sim$ (append(TailA, Y, TailC)), Head $\neq$ 1, $\sim$ (TailC = [2])

                    \item 
                        Aplicando a negação em TailC = [2]:

                        $\sim$ (append([Head|TailA], Y, [1, 2])) :- $\sim$ (append(TailA, Y, TailC)), Head $\neq$ 1, TailC $\neq$ [2]

                    \item 
                        Unificando a cláusula negada com a primeira cláusula do predicado append:
                        append([], Y, [1, 2])

                    \item 
                        Unificando a primeira cláusula com os termos correspondentes:

                        Y = [1, 2]

                        Contradição: com a unificação em VIII, temos Y igual a [1, 2], o que contradiz a negação em VI (TailC $\neq$ [2]).

                \end{enumerate}

            \item 
\begin{lstlisting}[language=prolog]
maxList([X], X).
maxList([X|Xs], M) :-
  maxList(Xs, Max),
  (X > Max -> M = X ; M = Max).
\end{lstlisting}

            \item 
\begin{lstlisting}[language=prolog]
nqueens(N, X) :-
  length(X, N),
  X ins 1..N,
  safe_queens(X).

safe_queens([]).
safe_queens([Q|Queens]) :-
  safe_queens(Queens, Q, 1),
  safe_queens(Queens).

safe_queens([], _, _).
safe_queens([Q|Queens], Q0, D0) :-
  Q0 #\= Q,
  abs(Q0 - Q) #\= D0,
  D1 #= D0 + 1,
  safe_queens(Queens, Q0, D1).
\end{lstlisting}
    \end{enumerate}
\end{document}
