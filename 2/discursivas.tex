\documentclass{article}

\usepackage{hyperref}               % Use hyperlinks
\usepackage{enumerate}
\usepackage{listings}
\usepackage{xcolor}

\usepackage[T1]{fontenc}            % Codificação para português 
\usepackage[portuguese]{babel}      % Português

\usepackage{hyphenat}               % Use hifens corretamente
\hyphenation{mate-mática recu-perar}

\usepackage{fancyhdr} % Header estilo DCC

\setlength{\headheight}{24pt} % O LaTeX reclama sem isso

\makeatletter
\renewcommand{\maketitle}{%
  \begin{center}
    {\LARGE\@title\par}
    \vskip 0.5em
    {\large\begin{tabular}[t]{c}\@author\end{tabular}\par}
    \vskip 1em
    {\large\textbf{\@date}\par}
    \vskip 1em
  \end{center}
}

% Define the WHILE language
\lstdefinelanguage{WHILE}
{
  morekeywords={while, do, end, if, then, else, mod},
  sensitive=true,
  morecomment=[l]{//},
  morecomment=[s]{/*}{*/},
  morestring=[b]',
  morestring=[b]",
}

% Set up the code block
\lstset{
  language=WHILE,
  basicstyle=\small\ttfamily,
  keywordstyle=\bfseries,
  commentstyle=\itshape\color{gray},
  stringstyle=\itshape\color{blue},
  showstringspaces=false,
  breaklines=true,
  frame=single,
  numbers=left,
  numberstyle=\tiny,
  numbersep=5pt,
  tabsize=2,
  escapechar=|,
}

\title{Universidade Federal de Minas Gerais

Ciência da Computação}
\author{Igor Lacerda Faria da Silva}
\date{Lista de Exercícios 2}

\begin{document}

\pagestyle{fancy}
\lhead{DCC024 Linguagens de Progração}
\rhead{2023.1}

\maketitle

\section*{Binding, escopo}

\begin{enumerate}
  \item 
    \begin{enumerate}
      \item Estático: 1
      \item Dinâmico: 2
    \end{enumerate}
  \item 
    \begin{enumerate}
      \item Existem 5 blocos: o bloco da função g, o bloco \textit{let...in} exterior, o bloco da função f, o bloco da função h e o seu \textit{let...in} interior.
      \item Os nomes definidos pelo programa são: g, x, inc, f, y, h, z.
      \item 
        \begin{itemize}
          \item g: bloco 1
          \item x: bloco 1
          \item inc: bloco 2
          \item f: bloco 2
          \item y: bloco 3
          \item h: bloco 2
          \item z: bloco 4
          \item inc: bloco 5 (shadow)
        \end{itemize}
      \item O valor de g 5 é 6. Se SML possuísse escopo dinâmico, seria 7. Esses valores seriam diferentes pois se o escopo fosse dinâmico, as atribuições seriam usadas conforme a execução do programa, ou seja, chegaria-se em h, que redefine inc, e essa nova definição que seria usada em f.
    \end{enumerate}
    \setcounter{enumi}{4}
  \item  
    \begin{enumerate}
      \item Apesar de correta, essa solução não é idiomática, pois não usa \textit{pattern matching}.
      \item Os caracteres de pipe não querem aparecer :(
        \begin{lstlisting}[language=ML]  
fun good_max (xs: int list) : int =
  case xs of
  [] => 0
  [x] => x
  x :: xs' =>
    let val max_rest = max xs'
    in if x > max_rest then x else max_rest
    end;
        \end{lstlisting}
    \end{enumerate}
  \setcounter{enumi}{6}
  \item  
    \begin{enumerate}
    \item O valor impresso é 7.
    \item
\begin{lstlisting}
fun expr () : int =
  let val x = 1
  in let val x = 2 in x + 1 end + let val y = x + 2 in y + 1 end
  end; 
\end{lstlisting}
  \end{enumerate}
\end{enumerate}

\end{document}
