\documentclass{article}

\usepackage{hyperref}               % Use hyperlinks

\usepackage[T1]{fontenc}            % Codificação para português 
\usepackage[portuguese]{babel}      % Português

\usepackage{hyphenat}               % Use hifens corretamente
\hyphenation{mate-mática recu-perar}

\usepackage{fancyhdr} % Header estilo DCC

\setlength{\headheight}{24pt} % O LaTeX reclama sem isso

\author{Igor Lacerda Faria da Silva}

\begin{document}

\pagestyle{fancy}
\lhead{Prof. Haniel Barbosa \\ Linguagens de Progração}
\rhead{DCC / ICEx / UFMG \\ 2023.1}

\begin{center}
    \textbf{Atividade - Lista 1}
\end{center}

\begin{enumerate}
\setcounter{enumi}{13}
  \item Ordem: da esquerda para a direita.
    \begin{itemize}
      \item Tipo 1, pois o par é do tipo (Inteiro, Real)
      \item  Primeiro: é usado o mesmo tipo de par de antes, (Inteiro, Real) (tipo 1). Segundo: como o resultado do primeiro é um Real, é gerado um par (Real, Inteiro) (tipo 3).
      \item Segundo: é avaliado antes do primeiro, resultando em um real (tipo 3). Primeiro: com o resultado, o cálculo atua sobre o par (Inteiro, Real), sendo do tipo 1.
      \item Primeiro: (Inteiro, Inteiro) $\mapsto$ Inteiro, tipo 2. Segundo: (Inteiro, Real), sendo o inteiro do par anterior, portanto tipo 1. Quarto: antece o terceiro pelo parêntes, (Real, Inteiro) $\mapsto$ Real, tipo 3. Terceiro: esquerda é real pelo segundo '+' e direita pelo quarto '+', logo, tipo 4.
    \end{itemize}
  \item 
    \begin{enumerate}
      \item Toda expressão em SML que possui um \texttt{if} precisa ter um \texttt{else}.
      \item Não é possível usar o operador \texttt{*} com tipos diferentes, \texttt{int} e \texttt{float}.
      \item Como o casamento dos padrões é feito de baixo pra cima, não é possível que o padrão do 0 seja atingido.
    \end{enumerate}
\end{enumerate}

\end{document}
